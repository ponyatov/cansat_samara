% e-book
\documentclass[oneside,10pt]{book}
%% screen paper layout (A5 landscape)
\usepackage[paperwidth=118.8mm,paperheight=68.2mm,margin=2mm]{geometry}
%% font setup for screen reading
\renewcommand{\familydefault}{\sfdefault}\normalfont
%% hyperlinks pdf style
\usepackage[unicode,colorlinks=true]{hyperref}
%% fix heading styles for tiny paper
\usepackage{titlesec}
\titleformat{\chapter}{\Large\bfseries}{\thechapter.}{1em}{}
\titleformat{\section}{\large\bfseries}{\thesection.}{1em}{}

% graphics
\usepackage[pdftex]{graphicx}
% \newcommand{\fig}[3]{\bigskip\noindent\includegraphics[#3]{#2}\textbf{#1}\bigskip}
\newcommand{\fig}[2]{\noindent\includegraphics[#2]{#1}}

% xcolor fixes
\usepackage{xcolor}
\definecolor{red}{rgb}{0.7,0,0}		% R
\definecolor{green}{rgb}{0,0.6,0}	% G
\definecolor{blue}{rgb}{0,0,0.7}	% B
\definecolor{darkblue}{rgb}{0,0,0.3}	% DB

% Cyrillization
%% \usepackage[T1,T2A]{fontenc}
\usepackage[utf8]{inputenc}
%% \usepackage[cp1251]{inputenc}
\usepackage[english,russian]{babel}
\usepackage{indentfirst}

% relative sectioning
\usepackage{ifthen}
\newcounter{secdepth}\setcounter{secdepth}{0}
\newcommand{\secup}{\addtocounter{secdepth}{1}}
\newcommand{\secdown}{\addtocounter{secdepth}{-1}}
\newcommand{\secrel}[1]{
\ifthenelse{\equal{\value{secdepth}}{0}}{\part{#1}}{}
\ifthenelse{\equal{\value{secdepth}}{-1}}{\chapter{#1}}{}
\ifthenelse{\equal{\value{secdepth}}{-2}}{\section{#1}}{}
\ifthenelse{\equal{\value{secdepth}}{-3}}{\subsection{#1}}{}
\ifthenelse{\equal{\value{secdepth}}{-4}}{\subsubsection{#1}}{}
}
\newcommand{\secly}[1]{\section*{#1} \addcontentsline{toc}{section}{#1} \refstepcounter{section}}
\newcommand{\subsecly}[1]{\subsection*{#1} \addcontentsline{toc}{subsection}{#1}  \refstepcounter{subsection}}

%% software menu & keys
\usepackage[os=win]{menukeys}
\usepackage{amssymb} % windows key
% \newcommand{\winstart}{$\boxplus$}
% \newcommand{\winr}{\keys{\winstart+R}}
\newcommand{\lms}{$\lhd$}
% \newcommand{\dblms}{$\lhd\lhd$}
\newcommand{\mms}{$\bigtriangleup$}
\newcommand{\rms}{$\rhd$}
% \newcommand{\checkbox}{$\boxtimes$}
% \newcommand{\uncheckbox}{$\square$}

%% listings
\usepackage{verbatim}
\usepackage{listings}
\lstset{
basicstyle=\small,
frame=single,
numbers=left,numberstyle=\small,numbersep=2mm,
tabsize=4,
keywordstyle=\textbf,
commentstyle=\color{blue}\textbf
}
\newcommand{\lst}[2]{\lstinputlisting[title=#2]{#1}}

%% languages
\newcommand{\win}{\texttt{Windows}}
\newcommand{\linux}{\texttt{Linux}}
\newcommand{\emlin}{\texttt{emLinux}}
\newcommand{\br}{\texttt{BuildRoot}}
\newcommand{\macos}{\texttt{MacOS}}
\newcommand{\mingw}{\texttt{MinGW}}
\newcommand{\ci}{$C$}
\newcommand{\cpp}{$C_+^+$}
\newcommand{\py}{\texttt{Python}}
\newcommand{\git}{\texttt{git}}
\newcommand{\rpi}{$R^\pi_4$}

%% program elements
\newcommand{\class}[1]{\textbf{\texttt{#1}}}
\newcommand{\fn}[1]{\texttt{#1}}
\newcommand{\var}[1]{\texttt{#1}}
\newcommand{\file}[1]{\texttt{#1}}

% [nosep] option in lists/enums
\usepackage{enumitem}
% frame box
\usepackage{framed}
% misc
\newcommand{\email}[1]{$<$\href{mailto:#1}{#1}$>$}
\renewcommand{\emph}[1]{\textcolor{blue}{#1}}
\newcommand{\term}[1]{\textcolor{green}{#1}}
\newcommand{\note}[1]{\,\footnote{\ #1}}
