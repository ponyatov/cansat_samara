\secrel{Установка}\label{pyinst}\secdown

\secrel{Виртуальная среда \py\ (venv)}

\noindent
ВС позволяет сделать копию интерпретатора \py\ \emph{отдельно для одного проекта}: со своим набором настроек и модулей (библиотек).

\begin{verbatim}
~/cansat_samara$ python3 -m venv ~/cansat_samara
~/cansat_samara$ . ~/cansat_samara/bin/activate
(venv)$ bin/pip3 install -U pip
(venv)$ bin/pip3 install -r requirements.txt
\end{verbatim}

\begin{description}[nosep]
    \item{\emph{pyserial}}  последовательный порт \ref{pyserial}
    \item{\emph{numpy}}     численные методы
    \item{\emph{graphviz}}  отрисовка графов \ref{graph}
    \item{\emph{ply}}       написание парсеров текстовых данных \ref{ply}
\end{description}

\clearpage
\secrel{Дополнительные пакеты}

При необходимости дополнительные пакеты устанавливаются вручную из командной строки:

\begin{verbatim}
~/cansat_samara$ . ~/cansat_samara/bin/activate
(venv)$ bin/pip3 install jupyter ...
\end{verbatim}

\begin{description}
    \item{\emph{jupyter}}   Jupyter Notebook: рабочая среда в браузере \ref{jupyter}
    \item{\emph{flask}}     Web-фреймворк и сервер для сетевых приложений \ref{flask}
\end{description}
    
\secup
