\secrel{Установка}\label{pyinst}\secdown

\begin{description}
    \item{типовой интерпретатор $CPython^3$} \url{https://www.python.org/downloads}
    \item{IDE среда разработки} \url{https://code.visualstudio.com/}
\end{description}

\begin{verbatim}
~/cansat_samara$ sudo apt install python3
\end{verbatim}
или \emph{полный комплект разработчика CanSat} (GNU/\linux):
\begin{verbatim}
~/cansat_samara$ sudo apt install -u `cat apt.txt`
\end{verbatim}

\noindent
После установки интерпретатора в \win\ прописать в \menu{Пуск>Компьютер>\rms>Параметры>Другие параметры}
значение переменной среды \var{PATH}:
\begin{verbatim}
PATH=C:\Python;C:\MinGW\bin;C:\MinGW\msys\1.0\bin;%PATH%
\end{verbatim}

\secrel{Установка виртуальной среды \py}

\begin{verbatim}
~/cansat_samara$ python3 -m venv ~/cansat_samara
~/cansat_samara$ . ~/cansat_samara/bin/activate
(venv)$ bin/pip3 install -U pip
(venv)$ bin/pip3 install -r requirements.txt
\end{verbatim}

\begin{description}
    \item{\emph{pyserial}}  последовательный порт \ref{pyserial}
    \item{\emph{numpy}}     численные методы
    \item{\emph{graphviz}}  отрисовка графов \ref{graph}
    \item{\emph{ply}}       написание парсеров текстовых данных \ref{ply}
\end{description}

\secrel{Дополнительные пакеты}

\begin{verbatim}
    ~/cansat_samara$ . ~/cansat_samara/bin/activate
(venv)$ bin/pip3 install jupyter ...
\end{verbatim}

\begin{description}
    \item{\emph{jupyter}}   Jupyter Notebook: рабочая среда в браузере \ref{jupyter}
    \item{\emph{flask}}     Web-фреймворк и сервер для сетевых приложений \ref{flask}
\end{description}
    
\secup
