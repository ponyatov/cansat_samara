\secrel{Спецификация системы CanSat}\label{can}\secdown

\url{http://www.esa.int/SPECIALS/CanSat/SEM6JVCKP6G_0.html}

\noindent
В зависимости от вида соревнований существуют требования к системам спуска и связи, 
радио CanSat должно быть совместимо с наземной станцией.

\bigskip
Вот некоторые общие спецификации CanSat для соревнований:

\begin{itemize}
\item Все компоненты должны соответствовать \emph{стандартным размерам}:
\begin{description}
    \item{класс \emph{CanSat}}\ \\ высота\ --- 115\,мм, диаметр\ --- 66\,мм
    \item{класс \emph{Open}}\ \\ высота\ --- 240\,мм, диаметр\ --- 146\,мм
\end{description}
\item \emph{Максимальная масса} ограничена:\\CanSat\ --- 0.35\,кг\,max, Open\ --- 1\,кг\,max
\item CanSat должет быть совместим с \term{системой запуска}
\item CanSat должен быть оборудован системой мягкой посадки (такой как парашют) чтобы быть повторно используемым после запуска
\item Антенны, передатчики, и любые другие выносные элементы конструкции не могут превышать установленный диаметр согласно классу,
до тех пор пока CanSat не завершит пусковой цикл
\item Выносные и отделяемые подсистемы, и система мягкой посадки  can exceed the length of the primary structure, up to a maximum length of 230 mm;
\item Explosives, detonators, pyrotechnics, flammable materials, dangerous materials and biological payloads are strictly forbidden. All materials used must be safe for personnel, equipments and the environment. MSDS will be requested in case of doubt;
\item CanSats shall operate off of battery or solar panels. The power must supply the systems at least during 1 hour;
\item Cansats can use RF communications. CanSats with RF Transmitters shall have properly licensed operators;
\item Depending on the competition the total cost of the CanSat cannot exceed a certain amount in some competitions is 1000 \euro.
\end{itemize}

\secup
